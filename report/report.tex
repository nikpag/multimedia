\documentclass[a4paper, 12pt]{article}

\input{/home/nick/latex-preambles/xelatex.tex}

\setmainfont{Minion Pro}

\newcommand{\imagesPath}{.}

\title{
	\textbf{Τεχνολογία Πολυμέσων} \\~\\
	Αναφορά Εξαμηνιαίας Εργασίας
}
\author{Νικόλαος Παγώνας, el18175}
\date{}

\begin{document}
\maketitle
	
\section{Εισαγωγή}

Η παρούσα αναφορά έχει σκοπό να δράσει συμπληρωματικά στο εξαμηνιαίο project του μαθήματος, δίνοντας επεξηγήσεις για τα βασικά στοιχεία της εφαρμογής. Η εφαρμογή είναι μία παραλλαγή της κρεμάλας (Hangman). Υλοποιήθηκαν όλες οι λειτουργικότητες των προδιαγραφών του project. Χρησιμοποιήθηκαν: η γλώσσα Java, το πλαίσιο JavaFX και η εφαρμογή SceneBuilder. \\

Ο κώδικας περιέχει εκτενή σχόλια για όλες τις κλάσεις, τις μεθόδους αλλά και τα πεδία τους, ώστε να γίνει κατά το δυνατόν κατανοητή η λειτουργία της εφαρμογής. Έγινε προσπάθεια ο κώδικας να ακολουθεί τις βασικές αρχές του αντικειμενοστρεφούς προγραμματισμού και η τεκμηρίωση να συμμορφώνεται με τις προδιαγραφές του εργαλείου \verb|javadoc|. \\

\section{Συνοπτική παρουσίαση του file directory και των συστατικών του πηγαίου κώδικα}
	
	\begin{itemize} 
		\item \textbf{images/} $\rightarrow$ Περιέχει τις εικόνες που αντιστοιχούν στο πλήθος των λανθασμένων επιλογών.
		\item \textbf{lib/} $\rightarrow$ Περιέχει τα \verb|.jar| αρχεία που απαιτούνται ώστε να μπορεί να λειτουργήσει η εφαρμογή. Αυτά είναι:
			\begin{itemize}
				\item Από το \verb|JavaFX SDK v11.0.2:|
				\begin{verbatim}
					javafx-swt.jar, javafx.base.jar, javafx.controls.jar, 
					javafx.fxml.jar, javafx.graphics.jar, javafx.media.jar, 
					javafx.swing.jar, javafx.web.jar
				\end{verbatim} 
				\item \verb|javax.json.jar| για διαχείριση \verb|JSON Objects|.
			\end{itemize} 
		
		\item \textbf{medialab/} $\rightarrow$ Περιέχει τα λεξικά που δημιουργούνται μέσω της εφαρμογής, όπως αναγράφεται και στις προδιαγραφές του project.
		
		\item \textbf{src/} $\rightarrow$ Περιέχει τον πηγαίο κώδικα της υλοποίησης, δηλαδή τις κλάσεις, τους controllers, τις εξαιρέσεις και την \verb|FXML| της εφαρμογής, τα οποία χωρίζονται στους παρακάτω υποφακέλους:
			
		\item \textbf{src/classes/} $\rightarrow$ Περιέχει όλες τις κλάσεις που δεν είναι controllers ή exceptions, δηλαδή την κύρια λογική της εφαρμογής. Αυτές είναι: 
				
			\begin{itemize}
				\item \verb|Dictionary|: Αναπαριστά ένα λεξικό. Περιέχει την λίστα των λέξεών του, καθώς και στατιστικά για αυτό.
				\item \verb|DictionaryCreator|: Δημιουργεί και αποθηκεύει ένα λεξικό καλώντας το Works API.
				\item \verb|Game|: Αναπαριστά μία παρτίδα και υλοποιεί την λογική του παιχνιδιού.
				\item \verb|Main|: Το entrypoint της εφαρμογής.
				\item \verb|PastGamesInfo|: Κρατάει στατιστικά στοιχεία για τα 5 τελευταία παιχνίδια (για να εμφανιστούν όταν ο χρήστης πατήσει \verb|Details/Rounds|).
				\item \verb|Tuple|: Αναπαριστά μία δυάδα της μορφής $(letter, probability)$, όπου $probability$ είναι η πιθανότητα το γράμμα $letter$ να είναι σωστή επιλογή. Η πιθανότητα αυτή υπολογίζεται όπως αναγράφεται στις προδιαγραφές του project. 
				\item \verb|TupleComparator|: Χρησιμοποιείται για τη σύγκριση δύο Tuples
			\end{itemize}
				
			\item \textbf{src/controllers/} $\rightarrow$ Περιέχει όλους τους controllers της εφαρμογής. Έχει δημιουργηθεί ένας controller ανά παράθυρο. Όλοι οι controllers υλοποιούν το Controller interface (που έχει επίσης οριστεί σε αυτόν τον φάκελο), το οποίο τους υποχρεώνει να ορίσουν μία μέθοδο \verb|setMainController()| που καθιστά δυνατή την επικοινωνία μεταξύ pop-up και main window. Οι Controllers είναι οι εξής:
				
				\begin{itemize}
					\item \verb|CreateController|: Υπεύθυνος για το pop-up window που δημιουργείται μόλις ο χρήστης πατήσει \verb|Application/Create| (\verb|create.fxml|). 
					\item \verb|DictionaryController|: Υπεύθυνος για το pop-up window που δημιουργείται μόλις ο χρήστης πατήσει \verb|Details/Dictionary| (\verb|dictionary.fxml|).
					\item \verb|LoadController|: Υπεύθυνος για το pop-up window που δημιουργείται μόλις ο χρήστης πατήσει \verb|Application/Load| (\verb|load.fxml|).
					\item \verb|MainController|: Υπεύθυνος για το main παράθυρο της εφαρμογής (\verb|main.fxml|).
					\item \verb|RoundsController|: Υπεύθυνος για το pop-up window που δημιουργείται μόλις ο χρήστης πατήσει \verb|Details/Rounds| (\verb|rounds.fxml|).
				\end{itemize}
				
			\item \textbf{src/exceptions/} $\rightarrow$ Περιέχει όλες τις custom εξαιρέσεις που χρειάζονται για το error handling της εφαρμογής. Αυτές είναι:
				
				\begin{itemize}
					\item \verb|InvalidCountException|
					\item \verb|InvalidRangeException|
					\item \verb|NoDescriptionFieldException|, που εγείρεται όταν κάποιο βιβλίο δεν έχει description.
					\item \verb|UnbalancedException|
					\item \verb|UndersizeException|
				\end{itemize}
				
			\item \textbf{src/fxml/} $\rightarrow$ Περιέχει όλα τα \verb|.fxml| αρχεία της εφαρμογής.
	\end{itemize}

\section{Παραδοχές}	

\begin{itemize}

	\item Επειδή κατά τη δημιουργία λεξικού φιλτράρουμε τα duplicates και τις λέξεις που έχουν λιγότερα από 6 γράμματα, στην πραγματικότητα τα InvalidCountException και InvalidRangeException δεν εγείρονται ποτέ. Παρολαυτά τα έχουμε ορίσει για λόγους πληρότητας.

	\item Τα στατιστικά των 5 τελευταίων παιχνιδιών διατηρούνται για όσο ο χρήστης έχει ανοικτή την εφαρμογή. Αν η εφαρμογή κλείσει, τότε τα στατιστικά χάνονται.

	\item Τα διαθέσιμα γράμματα για κάθε θέση εμφανίζονται ταξινομημένα στον χρήστη, ωστόσο δεν εμφανίζονται οι πιθανότητες τους, για να είναι πιο "καθαρή" η εμφάνιση. Κατά τα άλλα, οι πιθανότητες υπολογίζονται κανονικά.

	\item Στο Rounds, η πιο πρόσφατη παρτίδα βρίσκεται στο πάνω μέρος του παραθύρου και η λιγότερο πρόσφατη στο κάτω μέρος.

	\item Πέραν των βιβλίων που έχουν το \verb|description| σαν object, υποστηρίζονται \textbf{και} λεξικά που έχουν το \verb|description| σαν \verb|string| (δηλ. δεν έχουν πεδίο \verb|value| φωλιασμένο μέσα στο \verb|description|).

\end{itemize}

\textbf{Σημείωση:} Για να λειτουργήσει η εφαρμογή, πρέπει να οριστεί κατάλληλα η μεταβλητή \verb|Vm Args| (ανατρέξτε στις ρυθμίσεις του IDE της επιλογής σας), ώστε να περιλαμβάνει τα modules: 
\begin{verbatim}
	javafx.controls, javafx.fxml,
\end{verbatim}
όπως έχει επιδειχθεί και στην αντίστοιχη εργαστηριακή διάλεξη. Το \verb|JavaFX SDK v11.0.2| δεν συμπεριλαμβάνεται στον φάκελο της υποβολής και πρέπει να εγκατασταθεί ανεξάρτητα.
\end{document}